\chapter*{\Large \scshape S\.{i}mgeler ve Kisaltmalar}
\addcontentsline{toc}{section}{\footnotesize \scshape S\.\i mgeler ve Kısaltmalar}
\vspace{-0.35cm}
\begin{footnotesize}
\begin{center}
\end{center}
\vspace{-0.35cm}
\begin{flushleft}
\vspace{-1.0cm}\textbf{Simgeler}\\ \vspace{-0.3cm}\noindent\rule{\textwidth}{0.2pt}    
\end{flushleft}

\vspace{-0.7cm}



%----------------------------------------------------------------------------------------------------
%----------------------------------------------------------------------------------------------------
%	Simgiler ve kısaltmalar listesi  manuel olarak tanımlandığından dolayı bu listeyi kendiniz 
%	doldurmalınız. Yazım stili ve boşluklar üzerinde herhangi bir değişiklik yapmanız gerekmez.
%
%	Aşağıda tez yazım klavuzuna uygun formatta bir Simgeler ve Kısaltmalar Listesi bulunmaktadır.
%	Simge ve kısaltmalar listeniz 1 sayfadan fazla ise "ana.tex" dosyasında tanımlanmış olan "longtable" 
%	paketini kullanmanızı tavsiye ederiz. Eğer kendinize ait yeni bir simgeler ve kısaltmalar listesi 
%	oluşturmak istiyorsanız Enstitü tarafından yayınlanan TEZ YAZIM KILAVUZUNA UYMANIZ GEREKMEKTEDİR!
%----------------------------------------------------------------------------------------------------
%----------------------------------------------------------------------------------------------------
\begin{tabular}{p{0.14\textwidth}p{0.80\textwidth}}
\hspace{8.5cm}\\	

\textbf{b} 				&  	: Burgers vektörü\\
\textit{D}				&	: Tane boyutu\\
\textit{G}				&	: Gibbs serbest enerjisi\\
\textit{H}				&	: Entalpi\\
\textit{k}				&	: Hall-Petch sabiti\\
\textit{\(M_{s}\)}		&	: Martensit faz başlangıç sıcaklığı\\
\textit{\(r_{0}\)}		&	: Alt kesim mesafesi\\
\textit{\(r^{*}\)}		&	: Çekirdeklenme kritik yarıçapı\\
\textit{S}				&	: Entropi\\
\textit{T}				&	: Kelvin cinsinden sistem sıcaklığı\\
\textit{\textDelta \(G^{*}\)}	&	: Çekirdeklenme için aktivasyon enerjisi\\

\(\sigma_{y}\)			&	: Akma zoru\\
\(\tau\)				&	: Kesme zoru\\
\(\Theta\)				&	: Sapma açısı\\
\(\delta _{b}\)			&	: Tane sınır kalınlığı\\
\(\sum\)				&	: Tesadüfi-yer örgü karakteristiği\\
\end{tabular}

\vspace{1.5cm}											  

\begin{flushleft}
\vspace{-1.0cm}\textbf{Kısaltmalar}\\ \vspace{-0.3cm}\noindent\rule{\textwidth}{0.2pt}    
\end{flushleft}
\vspace{-0.7cm}

\begin{acronym}[MPC]
\acro{GSM}{Mobil iletişim için küresel sistem (Global System for Mobile Communications)}
\acro{IoT}{Internet of Things}
\acro{IoE}{Internet of Everythings}
\acro{3D}{Üç Boyutlu Uzay (Three-dimensional space)}
\acro{SOM}{Kendi Kendini Düzenleyen Harita (Self-Organizing Map)}
%GraphVis
\acro{PCA}{Ana Bileşen Analizi (Principal Component Analysis)}
\acro{MST}{Minimal Yayılma Ağacı (Minimal Spanning Tree)}
\acro{DFS}{Derinlik Öncelikli Arama (Depth First Search)}
\acro{BFS}{Genişlik Öncelikli Arama (Breadth First Search)}
\end{acronym}

\end{footnotesize}
\newpage
