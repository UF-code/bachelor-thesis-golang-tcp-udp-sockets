\chapter{G\.{i}r\.{i}ş}
\thispagestyle{empty}
\setcounter{page}{1}
\pagenumbering{arabic}

\justify
\begin{small}
\setlength{\parindent}{1cm}
\hspace{0.8cm}
%--------------------- Metin Başlangıcı ------------------------------------



% bu iki metin combine edilebilir

\vspace{30mm}
% \Large \textbf{Sonuç}
% \vspace{2mm}
Go dili 2007 de Google da çokçekirdekli, ağ bağlantılı makinelerde devasa kod bütünlüğüyle çalışabilmek ve programlama verimliliğini arttırabilmek adına dizayn edilmistir. Kasım 2009'da kamuya duyuruldu ve versiyon 1.0 Mart 2012'de yayınlandı. Go, Google'daki projelerde ve diğer birçok büyük kuruluşta ve açık kaynaklı projelerde yaygın olarak kullanılmaktadır. Go dili static yazılan bir dildir ve bu çalışma zamanının kısalmasını sağlar, okunulabilirlik ve kullanılabilirlik açısından kolay bir dildir aynı zamanda networking ve concurrent işlemlerde ise üstün bir performans sağlar.

Goroutines vs Threads, Go dilinde threadlerin yerine kullanabiliceğimiz
daha az kaynak tüketimi olan goroutine(sanal thread) ler bulunmaktadır. Her bir programın içerisinde en az bir goroutine bulunmaktadır ve bu goroutine ’main goroutine’ olarak adlandırılır.Diğer bütün goroutineler ’main goroutine’ in altında çalışırlar. Goroutineler ve dilin son derece modern yüksek seviye olması yazılım geliştirme sürecini kısaltıp etkili bir şekilde üretime geçilmesi Go dilinin en büyük avantajları arasındadır. 

Soketler veya 'Web Soketleri', bir server ile bir client veya daha fazla client arasında gerçek zamanlı ve kalıcı bağlantılar için TCP tabanlı bir internet protokolüdür. Bu protokol, standart HTTP bağlantılarından daha hızlıdır ve server ile clientlar arasında iki yönlü veri aktarımına izin verir. Bir handshake işlemi olarak bir HTTP yükseltme isteği ile başlatılır, ardından tüm taraflar, server ve clientlar herhangi bir zamanda, herhangi bir yönden, herhangi bir zamanda veri gönderebilir. Genellikle sosyal beslemeler, sohbet clientları, çok oyunculu oyunlar, multimedya akışı ve ortak geliştirme oturumları gibi uygulamalar için kullanılır.

Go dilindeki net paketi, OSI katmanlarından Uygulama ve Taşıma arasında soket iletişimini uygulamak için gerekli API ları sağlar. Net paketi, ağ I/O, TCP/IP, UDP, alan adı çözümlemesi ve Unix alan soketleri için bir arabirim sağlar. Bu paket ayrıca ağ temel öğelerine düşük düzeyde erişim sağlar. İletişim için temel arabirimler, Dial, Listen ve Accept gibi işlevler ve ilgili Conn ve Listener arabirimleri tarafından sağlanır.
\\

\end{small}